Our project, initially conceived as a mere class assignment, carries with it far-reaching implications that extend beyond its immediate scope. At first glance, one might dismiss its potential impact on climate, economics, and society, deeming it insignificant. After all, it's just a project, right? But the lens through which we are instructed to view it—imagining it as a commercial product—unveils a myriad of intriguing possibilities.

\subsection{Climate Impacts}

Exploring the multifaceted impact of our project, from its inception in the classroom to its hypothetical commercialization, unveils a vast landscape of possibilities and implications. While initially conceived as a simple academic exercise, the Better Than Elvis project transcends its humble origins, offering insights into the interconnectedness of technology, society, and the environment.

Beginning with climate impacts, the manufacturing process of the Better Than Elvis project is a significant point of consideration. Every component, from the microelectronics to the casing, entails resource extraction, processing, production, and transportation, with each step leaving an environmental footprint. The carbon emissions, energy consumption, and waste generated throughout the supply chain contribute to climate change, albeit to varying degrees. The more that are produced the larger this impact would be. Additionally, the materials chosen for the device are an important consideration. It is mostly made of plastic and metals, both things that biodegrade very slowly. During its life cycle, this is a good thing. We want our product to be lasting and reliable, two things that would be hard to do if it was constantly biodegrading. Products that are used for longer have less negative environmental impacts because not so many are being produced or thrown away. The problem is nothing lasts forever, and eventually (hopefully not in the near future) our product will either stop working or be so outdated that people stop using it. When that happens, it will probably get thrown away, and the materials it is made of will matter. Hopefully, it will be recycled, and this part of its environmental impact will be erased, but statistically speaking, a lot more things like this get thrown away than recycled.

The process for mitigating these environmental impacts does not stop at the manufacturing phase. Considerations must also extend to energy efficiency during operation, end-of-life recycling practices, and sustainable sourcing of materials. Ideas such as encouraging our customers to send us outdated boards and using recycled materials from them when we make newer ones could lower the strain this puts on the environment. By adopting a lifecycle approach to product development and management, we can minimize the environmental burden associated with our project, aligning it with principles of sustainability and responsible stewardship.

Of course, environmental impact is not all about minimizing negative impact. Maximizing the positive impact is essential as well. As such, the potential positive impacts of our project on climate should not be overlooked. By facilitating more efficient testing of electronic devices, our product could contribute to the development of energy-saving technologies, thereby indirectly reducing greenhouse gas emissions. Furthermore, in educational settings, it has the capacity to foster a deeper understanding of electronics and engineering principles, empowering future innovators to devise sustainable solutions to pressing environmental challenges. The future environmental impact of the inventions facilitated by this are potentially tremendous. There is no way to know how great the results can be of simply giving regular people access to cutting-edge testing equipment.

\subsection{Economic Impacts}

Transitioning to economic impacts, the affordability and accessibility of our product emerge as key drivers of change. Compared to existing alternatives, which may be prohibitively expensive for students and small-scale practitioners, our device offers a cost-effective solution without compromising on quality or functionality. This democratization of access to advanced technology levels the playing field, enabling individuals from diverse socioeconomic backgrounds to participate in innovation and entrepreneurship.

The economic ripple effects of such accessibility extend beyond individual users to encompass broader industry dynamics. By equipping aspiring engineers and innovators with the tools they need to succeed, our project has the potential to catalyze a wave of innovation and entrepreneurship. Startups and small businesses, empowered by affordable testing equipment, may develop groundbreaking technologies that disrupt traditional markets and create new avenues for economic growth.

The economic impacts of our project are not confined to the realm of entrepreneurship either. In schools, young engineers having access to the Better Than Elvis board can deepen their understanding of the electronics they are working with, making them better engineers coming out of school. This impacts all the companies they go to work for, potentially increasing productivity and innovation across companies new and old.

Its adoption within educational institutions, research laboratories, and industrial settings can generate economic value through enhanced productivity, efficiency, and competitiveness. By streamlining testing processes and accelerating product development cycles, our device enables organizations to achieve their goals more effectively, thereby driving economic growth and prosperity.

\subsection{Societal Impacts}

Turning our attention to societal impacts, the implications of our project extend far beyond the realm of technology and economics. At its core, our device represents a tool for knowledge dissemination and skill development, with far-reaching implications for education, workforce development, and social equity.

In educational settings, our device serves as a catalyst for experiential learning, enabling students to gain hands-on experience with electronics and engineering concepts. By bridging the gap between theory and practice, it enhances the effectiveness of STEM education and prepares students for careers in high-demand fields such as electronics and robotics.

Furthermore, by making advanced testing equipment more accessible to students from diverse backgrounds, our project promotes inclusivity and diversity within the STEM workforce. It makes hands-on learning more accessible to low-income demographics. By breaking down barriers to entry and providing opportunities for underrepresented groups to engage with cutting-edge technology, we contribute to the creation of a more equitable and inclusive society.

Beyond education, the societal impacts of our project extend to workforce development and economic empowerment. By equipping individuals with the skills and tools they need to succeed in the modern economy, we empower them to pursue meaningful careers and contribute to their communities' prosperity. Whether through entrepreneurship, innovation, or employment in established industries, the knowledge and experience gained from using our device can open doors to new opportunities and pathways to success.

Our project has the potential to make hands-on experience with electronics more accessible and exciting. This gives people an opportunity to experiment with electrical engineering before necessarily having to spend a lot on school or learn things that many have a negative association with, such as high-level math. Once they have experience with these things, it could make many more interested and excited and push more people into fields such as electrical engineering because they already know a little of what they are getting into.

By fostering a culture of innovation and collaboration, our project has the potential to catalyze broader social change. By bringing together diverse groups of individuals—students, educators, entrepreneurs, and industry professionals—we create opportunities for knowledge exchange, skill sharing, and collective problem-solving. In doing so, we not only advance the frontiers of technology but also strengthen the social fabric of our communities and societies.

In conclusion, the project transcends its origins as a classroom exercise to become a catalyst for change on multiple fronts. The actual implications of our product are impossible to entirely predict but, from its environmental implications to its economic and societal impacts, our project has the potential to have tremendous impacts on the future of technology, education, and innovation. By embracing principles of sustainability, accessibility, and inclusivity, we can maximize its positive contributions while mitigating potential negative consequences. As we continue to develop and refine our project, we remain mindful of its broader implications and strive to create a better, more equitable future for all.
